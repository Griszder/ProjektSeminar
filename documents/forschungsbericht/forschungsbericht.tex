\documentclass[paper=A4,pagesize=auto,12pt,headinclude=true,footinclude=true,BCOR=0mm,DIV=calc]{scrartcl}

\usepackage[utf8]{inputenc}
\usepackage{amsmath}
\usepackage[T1]{fontenc}
\usepackage{lmodern}
\usepackage[ngerman]{babel}
\usepackage[hidelinks]{hyperref}
\usepackage{setspace}
\usepackage[left=25mm, right=25mm, top=25mm, bottom=25mm]{geometry}
\usepackage[backend=biber, style=ieee, citestyle=ieee]{biblatex}
\usepackage[table]{xcolor}
\onehalfspacing

\bibliography{forschungsbericht}

%opening
\title{Auswirkungen eines Hintergrundes auf die Identifikation von Objekten in Bildern mithilfe eines Convolutional Neural Network}
\author{Alexandra Zarkh, Sui Yin Zhang,\\ Lennart Leggewie und Alexander Schallenberg}

\makeatletter
\def\@maketitle{%
	\newpage
	\null
	\vskip 2em%
	\begin{center}%
		\let \footnote \thanks
		{\Huge \textbf{\@title} \par}%
		\vskip 3em%
		{\large
			\lineskip .5em%
			\begin{tabular}[t]{c}%
				\@author
			\end{tabular}\par}%
		\vskip 2em%
		{\large Hochschule Bonn-Rhein-Sieg, Fachbereich Informatik, D-53757\par}%
		\vskip 2em%
		{\large \@date}%
	\end{center}%
	\par
	\vskip 1.5em}
\makeatother


\setlength{\arrayrulewidth}{0.3mm}
\setlength{\tabcolsep}{10pt}

\newcommand{\equations}{
\rowcolors{2}{white}{lightgray}
\begin{equation}
	\begin{tabular}{|c|c|}
		\hline
		\multicolumn{2}{|c|}{\textbf{Schildbilder im Trainingsdatensatz}}\\ \hline \hline
		\textbf{Andreaskreuz} & 5 \\ \hline
		\textbf{Fußgängerüberweg} & 6 \\ \hline
		\textbf{Gefahrenstelle} & 2 \\ \hline
		\textbf{Vorfahrt gewähren} & 5 \\ \hline
		\textbf{Vorfahrtsstraße} & 6 \\ \hline
		\textbf{Verbot der Einfahrt} & 8 \\ \hline
		\textbf{Fußgängerweg} & 8 \\ \hline
		\textbf{Stopp} & 4 \\ \hline
	\end{tabular}
\end{equation}
\vspace{2em}
\begin{equation}
	\begin{tabular}{|c|c|c|c|c|c|}
		\hline
		\multicolumn{6}{|c|}{\textbf{Anzahl Neuronen im CNN}}\\ \hline \hline
		\textbf{Input} & \textbf{Hidden 1} & \textbf{Hidden 2} & \textbf{Hidden 3} & \textbf{Hidden 4} & \textbf{Output}\\ \hline
		4096 & 64 & 64 & 64 & 64 & 8\\ \hline
	\end{tabular}
\end{equation}
\vspace{2em}
\begin{equation}
	\begin{tabular}{|c|c|}
		\hline
		\multicolumn{2}{|c|}{\textbf{Bildgröße}}\\ \hline \hline
		\textbf{Höhe} & \textbf{Breite}\\ \hline
		64 & 64\\ \hline
	\end{tabular}
\end{equation}
\vspace{2em}
\begin{equation}
	\begin{tabular}{|c|c|}
		\hline
		\multicolumn{2}{|c|}{\textbf{Training}}\\ \hline \hline
		\textbf{Lernrate} & \textbf{Wiederholungen}\\ \hline
		0.12 & 100000\\ \hline
	\end{tabular}
\end{equation}
}



\begin{document}

\begin{titlepage}
	\maketitle
\end{titlepage}

\tableofcontents
\newpage


\section{Zusammenfassung}


\section{Einleitung}
%Alexi

\subsection{Verwendete Literatur}

\subsection{Theorie} %Alexi

\subsection{Fragestellung}
Die Hauptfrage ist nun: Welche Auswirkungen hat ein Hintergrund eines Bildes auf die Kosten der Kalkulation eines Convolutional Neural Network (CNN), welches ein Objekt im Vordergrund des Bildes erkennen und identifizieren soll?


\section{Methoden}

\subsection{Convolutional Neural Networks} %Lennart

\subsection{Implementation}
%Sui Yin

\subsubsection{ConvNet} %Alex
Die Klasse \textit{ConvNet} soll ein Convolutional Neural Network darstellen und erbt daher von der im Projektbericht \cite{projektbericht} unter Abschnitt 2 genannten Klasse \textit{Network}, was jener die gleichen Eigenschaften verleiht. Des Weiteren besitzt sie eine \textit{java.util.ArrayList} vom Generic-Typ \textit{TrainData} und einen \textit{ImageAdapter} als private konstante Attribute. Erstere dient dem Speichern von Trainingsdatensätzen, die dann mit dem Aufrufen der überladenen Methode \textit{train(double, int)} benutzt werden. Diese ruft die originale Instanz-Methode \textit{train(double[][], double[][], double, int)} der Klasse \textit{Network} mit den Daten aus der \textit{ArrayList} auf.
Außerdem bietet die Klasse \textit{ConvNet} die Instanz-Methode \textit{addTrainData(String, int[])} um dieser Trainingsdatensätze hinzuzufügen. Diese nimmt einen Dateinamen von einem Bild und einen ganzzahligen Zielvektor. Zwei private nicht-statische Hilfsmethoden \textit{getImages()} und \textit{getTargets()} dienen der besseren Handhabung der \textit{ArrayList}.\\
Die Klasse erstellt durch den Konstruktor \textit{ConvNet(String, int, int, int, int...)} (Pfad zum Ordner mit den Bildern, gewünschte Bildweite, -höhe, Anzahl der Output-Neuronen, Anzahl der Hiddenlayers und deren Neuronen) von ihr abgeleitete Objekte.

\subsubsection{ImageAdapter} %Sui Yin

\subsubsection{TrainData}
Das Record \textit{TrainData(int[], int[])} besitzt einen Vektor mit den RGB-Werten eines Bildes und einen Zielvektor. Außerdem hat es zwei Instanz-Methoden \textit{getImageAsDoubleArray()} und \textit{getTargetAsDoubleArray()}, welche die jeweiligen ganzzahligen Vektoren als Gleitkommazahl-Vektoren zurückgibt, und eine private nicht-statische Hilfsmethode \textit{getIntAsDoubleArray(int[])}.

\subsubsection{Test-Dateien} %Alex
Für das Testen wurden ein Enum \textit{RoadSignLabel} und eine Klasse \textit{RoadSignTest} implementiert. Das Enum zählt die möglichen Ergebnisse des Netzes auf, indiziert sie und kann den jeweils richtigen Zielvektor durch die Instanz-Methode \textit{getTarget()} zurückgeben. In der Klasse befindet sich die für die Forschungsfrage relevante Main-Methode des Programms. Diese ruft eine private statische Methode \textit{test()} in der selben Klasse auf. In dieser wird ein \textit{ConvNet} deklariert und initialisiert, mit Datensätzen bestückt und trainiert. Dabei werden einige hilfreiche Ausgaben gemacht.

\subsection{Tests}


\section{Ergebnisse}


\section{Grafiken \& Tabellen}
\equations

\section{Diskussion}


\printbibliography[heading=bibnumbered]


\section{Anhang}



\end{document}
