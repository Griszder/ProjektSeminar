\documentclass[paper=A4,pagesize=auto,12pt,headinclude=true,footinclude=true,BCOR=0mm,DIV=calc]{scrartcl}

\usepackage[latin1]{inputenc}
\usepackage[T1]{fontenc}
\usepackage{lmodern}
% neue deutsche Trennungsregeln, etc
\usepackage[ngerman]{babel}
\usepackage[hidelinks]{hyperref}
\usepackage{setspace}	% \onehalfspaceing
% Seitenränder
\usepackage[left=25mm, right=25mm, top=25mm, bottom=25mm]{geometry}
% 1,5 Zeilenabstand
\onehalfspacing

%\usepackage[scaled]{helvet}

%opening
\title{Neuronale Netze}
\author{Alexandra Zarkh, Sui Yin Zhang,\\ Lennart Leggewie, Alexander Schallenberg}

\makeatletter
\def\@maketitle{%
	\newpage
	\null
	\vskip 2em%
	\begin{center}%
		\let \footnote \thanks
		{\Huge \textbf{\@title} \par}%
		\vskip 1em%
		{\Large \textbf{Projektbericht}\par}%
		\vskip 2em%
		{\large
			\lineskip .5em%
			\begin{tabular}[t]{c}%
				\@author
			\end{tabular}\par}%
		\vskip 2em%
		{\large Hochschule Bonn-Rhein-Sieg\par}%
		\vskip 2em%
		{\large \@date}%
	\end{center}%
	\par
	\vskip 1.5em}
\makeatother

\begin{document}

\begin{titlepage}
	\maketitle
\end{titlepage}

\tableofcontents
\newpage

\section{Einleitung}

\subsection{Aufgabenstellung \& Zielsetzung}
Die Aufgabenstellung lautete zun�chst, die Grundlagen eines k�nstlichen neuronalen Netzes (knN) in Java zu implementieren, sodass mit diesem schon zu einfachen Eingaben eine korrekte Ausgabe kalkuliert wird (Forward Propagation). Diese weiteten sich darauf aus, das Netz trainieren zu k�nnen (Backpropagation) und die trainierten Einstellungen des Netzes zu speichern und zu laden.\\
Das Ziel war, die erste H�lfte der Aufgabenstellung in den ersten zwei Wochen und die zweite H�lfte in den folgenden zwei Wochen umzusetzen.

\subsection{Aufgabenkontext \& externe Vorgaben}
Vorgegeben war, ein knN erstellen zu k�nnen, dem man bei seiner Erstellung Gewichtungen sowie Biases �bergeben kann. Au�erdem soll das Netz Ausgaben abh�ngig von den Eingaben berechnen k�nnen und die Gewichtungen und Biases sollen trainiert werden k�nnen. Au�erdem ist ein Format zur Abspeicherung der Gewichte und Biases vorgegeben worden, welche in diesem Format in eine CSV-Datei gespeichert werden soll.

\subsection{Nicht vorgegeben aber notwendigerweise von uns festgelegt}
Nicht vorgegeben, aber notwendigerweise festgelegt wurde, dass beim Erstellen eines knN die Anzahl der Neuronen f�r jede Neuronenschicht vom Benutzer festgelegt wird. Implizit wird damit auch die Anzahl der versteckten Schichten verlangt. Au�erdem ist die Struktur des Netzes frei gew�hlt.
% ggf. functions

\subsection{Literaturarbeit: Verweis auf Vorarbeiten}
\begin{itemize}
	\item{An Introduction to Neural Networks, Kroese, B., a Van der Smagt, P., 1996}
	\item{Neural Networks, 3Blue1Brown, 2018, \hyperref{https://www.youtube.com/playlist?list=PLZHQObOWTQDNU6R1_67000Dx_ZCJB-3pi}{}{}{YouTube}}
\end{itemize}


\newpage

\section{Methoden} % Aufteilen, wer was macht
% Beschreibung \& Begr�ndung der Umsetzung
% Beschreibung der Feinstrukturen (Programmteile, Objekte/Klassen und besondere Herausforderungen)
\subsection{Util}
Die Klasse \textit{Util} bietet die abstrakten Hilfsmethoden \textit{random(int)}, \textit{addToVec1(double[], double[])} und \textit{mulToVec(double, double[])} f�r das kreieren und f�r den Umgang mit Vektoren bzw. Arrays. Da diese Methoden mit ausreichend JavaDoc ausgestattet sind, wodurch sie selbsterkl�rend sind, wird auf diese hier nicht n�her eingegangen.

\subsection{Klassenstruktur Network \& Neuron}

\subsection{Konstruktoren \& Initialisierung}

\subsection{Kalkulation}
\subsubsection{Forward Propagation}

\subsection{Training}
\subsubsection{Backpropagation}

\subsection{Speichern \& Laden}

\subsection{toString}�

\newpage

\section{Ergebnisse}

\subsection{Begr�ndung der Korrektheit der Umsetzung}

\subsection{Performance-�berlegungen}

\newpage

\section{Diskussion \& Fazit}

\subsection{Vor- \& Nachteile der gew�hlten Umsetzung}

\subsection{Was fehlt ? Was k�nnte erweiternd gemacht werden?}

\newpage

\section{Verwendete Literatur \& Anhang}

\end{document}
